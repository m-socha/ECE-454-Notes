\documentclass[12pt,titlepage]{article}
\usepackage[margin=1in]{geometry}
\usepackage{parskip}

\usepackage{hyperref}
\hypersetup{
  linktoc=all
}

\begin{document}
  \begin{titlepage}
    \vspace*{\fill}
    \centering

    \textbf{\Huge ECE 454 Course Notes} \\ [0.4em]
    \textbf{\Large Distributed Computing} \\ [1em]
    \textbf{\Large Michael Socha} \\ [1em]
    \textbf{\large 4A Software Engineering} \\
    \textbf{\large University of Waterloo} \\
    \textbf{\large Spring 2018} \\
    \vspace*{\fill}
  \end{titlepage}

  \newpage 

  \tableofcontents

  \newpage

  \section{Course Overview}
    \subsection{Logistics}
      \begin{itemize}
        \item \textbf{Professor:} Wojciech Golab
      \end{itemize}

  \section{Introduction}
    \subsection{What is a Distributed System?}
      A distributed system is a collection of autonomous computing elements that appear to its users to be a single coherent system.

    \subsection{Rationale Behind Distributed Systems}
      \begin{itemize}
        \item Resource sharing
        \item Integrating multiple systems
        \item Centralized systems may not be as effective as many smaller systems working together
        \item Users themselves may be mobile and distributed physically from one another
      \end{itemize}

    \subsection{Middleware}
      Middleware is a layer of software that separates applications from their underlying platform. Middleware is often used by distributed
      systems to support heterogeneous computers and networks while offering a single-system view. Common middleware services include:
      \begin{itemize}
        \item Communication (e.g. add job to remote queue)
        \item Transactions (e.g. access multiple independent services atomically)
        \item Service composition (e.g. Independent map system enhanced with independent weather forecast system)
        \item Reliability (e.g. replicated state machine)
      \end{itemize}

    \subsection{Goals of Distributed Systems}
      \subsubsection{Supporting Resource Sharing}
        Resources can include:
        \begin{itemize}
          \item Peripheral devices (e.g. printers, video cameras)
          \item Storage facilities (e.g. file servers)
          \item Enterprise data (e.g. payroll)
          \item Web page (e.g. Web search)
          \item CPUs (e.g. Supercomputers)
        \end{itemize}

      \subsubsection{Making distribution transparent}
        Distribution transparency attempts to hide that processes and resources are physically distributed. Types of transparency include:
        \begin{itemize}
          \item \textbf{Access:} Hide differences in data representation and how data is accessed
          \item \textbf{Location:} Hide where resource is located
          \item \textbf{Migration:} Hide that a resource may move to another location
          \item \textbf{Relocation:} Hide that a resource may move to another location while in use
          \item \textbf{Replication:} Hide that a resource is replicated
          \item \textbf{Concurrency:} Hide that a resource may be shared by users competing for that resource
          \item \textbf{Failure:} Hide the failure and recovery of a resource
        \end{itemize}

      \subsubsection{Being Open}
        An open distributed system offers components that can be easily used by or integrated into other systems. Key properties of openness
        include:
        \begin{itemize}
          \item Interoperability
          \item Composability
          \item Extensibility
          \item Separation of policy from mechanism
        \end{itemize}

      \subsubsection{Being Scalable}
        Scalability covers a system's ability to expand along three axes:
        \begin{itemize}
          \item Size (e.g. adding users and resources)
          \item Geography (e.g. users across large distances)
          \item Administration (e.g. multiple independent admins)
        \end{itemize}
        Centralized systems tend to have limited scalability.

        A few common scaling techniques include:
        \begin{itemize}
          \item Hiding communication latencies (i.e. trying to avoid waiting for responses from remote-server machines)
          \item Partitioning (i.e. Taking a component, splitting it into smaller parts, and spreading those parts across the system)
          \item Replication (i.e. Adding multiple copies of a component to increase availability, balance load, support caching, etc.)
        \end{itemize}

      \subsection{Common Fallacies of Distributed Computing}
        The following are common beginner assumptions that can lead to major trouble:
        \begin{itemize}
          \item The network is reliable
          \item The network is secure
          \item The network is homogeneous
          \item The network topology is static
          \item Latency is 0
          \item Bandwidth is unlimited
          \item Transport cost is 0
          \item There is one administrator
        \end{itemize}

      \subsection{Types of Distributed Systems}
        \begin{itemize}
          \item \textbf{Web sites and Web services}
          \item \textbf{High performance computing (HPC):} High-performance computing in distributed memory settings, where message communication is used instead of shared memory.
          \item \textbf{Clustered Computing:} Distributing CPU or I/O intensive tasks over multiple servers.
          \item \textbf{Cloud and Grid Computing:} Grid computing focuses on combining resources from different institutions, while cloud computing provides access to shared pools
          of configurable system resources.
          \item \textbf{Transaction Processing:} Distributed transactions coordinated by transaction processing (TP) monitor.
          \item \textbf{Enterprise Application Integration (EAI):} Middleware often used as communication facilitator in such systems.
          \item \textbf{Sensor Networks:} Each sensor in a network can process and store data, which can be queries by some operator. Sensors often rely on in-network data
          processing to reduce communication costs.
        \end{itemize}

  \section{Architectures}
    \subsection{Definitions}
      \begin{itemize}
        \item \textbf{Component:} A modular unit with a well-defined interface
        \item \textbf{Connector:} Mechanism that mediated communication, coordination, or cooperation among components
        \item \textbf{Software architecture:} Organization of software components
        \item \textbf{System architecture:} Instantiation of software architecture in which software components are placed on real machines
        \item \textbf{Autonomic System:} System that adapts to its environment by monitoring its own behavior and reacting accordingly
      \end{itemize}

    \subsection{Architectural Styles}
      \subsubsection{Layered}
        Control flows from layer-to-layer; requests flow down the hierarchy and responses flow up, with each layer only interacting with its two neighboring layers.
        Layered architectures are often used to support client-server interactions, where a client component requests a service from a server component and waits for
        a response. Many enterprise systems are organized intro three layers, namely a user interface, application layer, and database. Note that middle layers may
        act as a client to the layer below and a server to the layer above.

        Vertical distribution describes the logical layers of a system being organized as separate physical tiers. Horizontal distribution describes a single logical
        layer being split across multiple machines.

      \subsubsection{Object-Based Architecture}
        Components are much more loosely organized than in a layered architecture, with components being able to interact freely with one another and no strict concept
        of layer.

      \subsubsection{Data-Centered Architecture}
        Components communicate by using a shared data repository, such as a database or file system.

      \subsubsection{Event-Based Architecture}
        Components communicate by sharing events. Publish/subscribe systems can be used for sharing news, balancing workloads, asynchronous workflows, etc.\\

      In practice, many systems are hybrids of the architectures listed above.

    \subsection{Self-Management:}
      Self-managing systems can be constructed using a feedback loop that monitors system behaviors and adjusts the system's internal operation.

  \section{Processes}
    \subsection{Controlling Processes in Linux}
      Below are some useful command:
      \begin{itemize}
        \item ps - lists running processes
        \item top - lists processes with top resource usage
        \item kill / pkill / killall - used to terminate processes
        \item jobs - lists currently running jobs
        \item bg - backgrounds a job
        \item fg - foregrounds a job
        \item nice / renice - sets priority of processes
        \item CTRL-C - stops job in terminal
        \item CTRL-Z - suspends job in terminal (use fg or bg to resume)
      \end{itemize}

    \subsection{Context Switching During IPC}
      Inter-process communication can be used to help coordinate processes. However, IPS can be costly, since it requires a context switch from user to
      kernel space and back.

    \subsection{Threads}
      Threads execution the same process can communicate through shared memory. Threads are sometimes referred to as lightweight processes (LWPs).
      Multithreading applications often follow a dispatcher/worker design, where a dispatcher thread received requests and feeds them to a pool
      of worker threads.

    \subsection{Hardware and Software Interfaces}
      Processes and threads may interact with underlying hardware either directly through processor instructions or indirectly through library functions
      operating system calls. The general layers of interaction (from more abstract to less) are applications, libraries, operating system and hardware.
      These layers may interact with any layer below. Distributed systems often run in virtual environments that manage interactions with lower layers.
      A major benefit of such virtualization is improved portability. Virtual machine monitors can offer additional benefits, including server consolidation,
      live migration of VMs to support load balancing and proactive maintenance, and VM replication.

    \subsection{Interfaces in Network Systems}
      Networked applications communicate by exchanging messages. The format of these messages is determined by a message-passing protocol.

    \subsection{Server Clusters}
      Servers in a cluster are often organized in three physical tiers. The first is a load balancer, followed by application servers followed by a database
      or file system.

  \section{Communication}
    \subsection{Access Transparency}
      Middleware can be used to provide access transparency for a distributed system, meaning that differences in data representation and how data is accessed
      can be hidden. This is typically done by middleware isolating the application layer from the transport layer.

    \subsection{Remote Procedure Calls (RPCs)}
      RPCs serve as the equivalent of conventional procedure calls, but for distributed systems. RPCs are implemented using a client-server protocol, where
      a client interacts with a network using a piece of software known as a client stub, and a server interacts with a network using a server stub. The
      steps of a typically RPC are detailed below:
      \begin{enumerate}
        \item Client process invokes client stub using ordinary procedure call.
        \item Client stub builds message, passes it to client OS.
        \item Client OS sends message to server OS.
        \item Server OS delivers message to server stub.
        \item Server stub unpacks parameters, invokes appropriate handler in server process.
        \item Handler processes message, returns result to server stub.
        \item Server stub packs result into message, passes it to server OS.
        \item Server OS sends message to client OS.
        \item Client OS delivers message to client stub.
        \item Client stub unpacks result, delivers it to client process.
      \end{enumerate}

      \subsubsection{Representation of Values}
        Packing parameters into a message is known as parameters marshalling (with unpacking known as demarshalling). Although data representations can be
        different across multiple machines (e.g. big-endian vs little-endian representation), the purpose of marshalling is to represent data in a
        machine-independent and network-independent format that all communicating parties expect to receive. The signatures of RPC calls can be defined
        using what is known as an interface definition language (IDL).

      \subsubsection{Synchronous vs Asynchronous}
        In a synchronous RPC, the client waits for a return value from the server before resuming execution. In an asynchronous RPC, the client may resume
        execution as soon as the server acknowledges receipt of the request (client does not need to wait for a return value). A variation of an asynchronous
        request where a client does not wait to receive any acknowledgment from the server is known as a one-way RPC.

    \subsection{Message Queuing Model} 
      As an alternative to RPCs, components in a distributed system may also communicate using a message queue that persists messages until they are consumed
      by a receiver. This technique allows for persistent communication that does not need to be tightly coupled in time. The primitive actions of a simple
      message queue are:
      \begin{itemize}
        \item \textbf{Put:} Append message to queue.
        \item \textbf{Get:} Block until queue not empty, then remove first message.
        \item \textbf{Poll:} Check specified queue for messages, remove the first message. Never block.
        \item \textbf{Notify:} Install handler to be called when message put into queue.
      \end{itemize}

      A key disadvantage of message queuing is that the delivery of the message ultimately rests with the receiver, and often cannot be guaranteed. Message
      queuing follows a design similar to publish-subscribe architectures, and is an example of message-oriented middleware (MOM).

    \subsection{Coupling Between Communicating Processes}
      \subsubsection{Referential Coupling}
        Referential coupling means that one process has to explicitly reference another one for them to communicate (e.g. connect to web server using IP
        address and port number).
      \subsubsection{Temporal Coupling}
        Temporal coupling means that processes must both be running for them to communicate (e.g. client cannot execute RPC if server is down).

  \section{Apache Thrift}
    \subsection{Overview}
      Apache Thrift is an IDL first developed by Facebook, and now managed as an open-source project in the Apache Software Foundation. Thrift provides
      a software stack and code generation engine to support RPCs between applications written in a wide variety of common languages, including C++, Java,
      Python, and PHP.

    \subsection{Thrift Software Stack}
      The elements in the Thrift software stack, from top to bottom, as as follows:
      \begin{itemize}
        \item \textbf{Server:} Sets up the lower layers of the stack, and then awaits incoming connections, which it hands off to the processor. Servers
        can be single or multi-threaded.
        \item \textbf{Processor:} Handles reading and writing IO streams, and is generated by the Thrift compiler.
        \item \textbf{Protocol:} Defined mechanism to map in-memory data structures to wire format (e.g. JSON, XML, compact binary)
        \item \textbf{Transport:} Handles reading to and writing from network (e.g. using HTTP, raw TCP, etc)
      \end{itemize}

    \subsection{Distribution Transparency}
      If Thrift clients must know the hostname and port for a given service, location transparency is violated. Also, although IDLs seek to provide access
      transparency, that too may be violated in certain conditions (e.g. when certain protocol or transport exceptions are thrown).

    \subsection{Application Protocol Versioning}
      Thrift fields can be modified and remain compatible with old versions, provided that the rules below are followed:
      \begin{itemize}
        \item Fields are associated with tag numbers, which should never the changed.
        \item New fields must be optional and provide default values.
        \item Fields no longer needed can be removed, but their tag numbers cannot be reused.
      \end{itemize}

    \subsection{Some Programming Tips}
      \begin{itemize}
        \item To separate policy from mechanism, it is generally a bad idea to hardcode hostnames and port numbers; it is usually preferable to accept command line
        arguments or use a property file.
        \item Objects built on top of TCP/IP connections (e.g. TSocket, TServerSocket) should be reused if possible to avoid to overhead of establishing and tearing
        down connections.
        \item By default Thrift transports, protocols and client stubs are not thread safe; different threads should share these items carefully, or not share them
        at all.
      \end{itemize}

  \section{Distributed File Systems}

    \subsection{Modes of Access}
      \begin{itemize}
        \item \textbf{Remote access model:} Client sends requests to access file stored on remote server.
        \item \textbf{Upload/download model:} File is downloaded from remote server, processes on client, and uploaded back to the server.
      \end{itemize}

    \subsection{Network File System (NFS)}
      NFS is a DFS first developed by Sun Microsystems in 1984, and remains heavily used in Unix-like systems. Features include:
      \begin{itemize}
        \item Client-side caching to reduce client-server communication
        \item Delegation of files from a server to a client, where a client can temporarily store and access a file, after which the delegation is recalled and
        the file returned.
        \item Compound procedures (e.g. lookup file, open file, and read data all in one call instead of 3). Note that NFS uses RPCs internally.
        \item Exporting different parts of a file system to different remote servers.
      \end{itemize}

    \subsection{Google File System (GFS)}
      GFS is a DFS that stripes files across multiple commodity servers, and is layered on top of ordinary Linux file systems. Although GFS is proprietary, the
      Hadoop Distributed File System is a simplified open-source implementation of GFS.
      \begin{itemize}
        \item \textbf{Synchronization:} A GFS master stores data about files and chunks. This metadata is cached in the main memory of chunk servers, and updates to
        these chunks servers are written to their own main memory. The master periodically polls the chunk servers to keep this metadata consistent.
        \item \textbf{Reads:} For reads, a client sends a file name and chunk index to the master, which responds with an address for the server storing that chunk.
        \item \textbf{Writes:} Updates are written directly to chunk servers, after which the changes are propagated through all primary and then secondary replicas.
      \end{itemize}

    \subsection{File Sharing Semantics}
      In centralized file systems, reads and writes are strictly ordered in time. This ensures that an application can always read its own writes. This behavior is
      often described as UNIX semantics, and can be attained in a DFS so long as there is a single file server and files are not cached.

      When a cached file in a DFS is modified, it is impractical to immediately propagate the changes back to the remote server (this would largely defeat the purpose
      or caching in the first place). Instead, a widely implemented rule is that changes are only visible to the process or machine that modified a file, and only made
      visible to other processes or machines when the file is closed. This rule is known as session semantics.

      Semantics of DFS can be defined using combinations of various techniques. For example, NFSv4 supports session semantics and byte range file locking, and HDFS
      provides immutable files along with an append function (e.g. for storing log data).

  \section{Hadoop MapReduce}

    \subsection{Background}
      MapReduce frameworks address the problem of parallelizing computations on big data sets across many machines. Google did much of the initial work on a generic
      MapReduce framework intended to phase out the need for special-purpose frameworks for different kinds of computations. Hadoop MapReduce is the most prominent
      open-source implementation of Google's MapReduce framework, and was created by Yahoo engineers Doug Cutting and Mike Cafarella.

    \subsection{High-level Architecture}
      Hadoop consists of a MapReduce engine and an HDFS system. The MapReduce layer contains a JobTracker, to which clients can submit MapReduce jobs. This JobTracker
      pushes work to available TaskTracker nodes, with the goal of picking a node close to the data that job requires.

    \subsection{MapReduce Basics}
      A few key guiding principles of MapReduce are:
      \begin{itemize}
        \item Components do not share data arbitrarily, as the communication overhead for keeping data synchronized across nodes can be very high.
        \item Data elements are immutable. Computation occurs by generating new outputs, which can then be forwarded into the next computation phase.
        \item MapReduce transforms lists of input data elements into lists of output data elements. A single MapReduce program typically does so twice, once for the
        Map phase and once for the Reduce phase.
      \end{itemize}

      \subsubsection{Mappers}
        A list of data elements are provided to a mapper one-by-one, which transforms each element to some output element.

      \subsubsection{Reducers}
        A reducer receives an iterator of input values, and combined these values together to produce a single value.

    \subsection{Technical Definitions}
      \begin{itemize}
        \item \textbf{InputSplit:} A unit of work assigned to a single map task.
        \item \textbf{InputFormat:} Determines how input files are parsed, which also determined the InputSplit.
        \item \textbf{RecordReader:} Loads data from input split, creating key-value pairs used by the mapper.
        \item \textbf{Partitioner:} Determines which partition key-values pairs should go to.
        \item \textbf{OutputFormatter:} Determines how output files are formatted.
        \item \textbf{RecordWriter:} Writes records to output files.
      \end{itemize}

    \subsection{Fault Tolerance}
      The main way Hadoop achieves fault tolerance is by restarting failed tasks. TaskTracker nodes are in constant communication with the JobTracker node, so should
      a TaskTracker fail to respond in a given period of time, the JobTracker will assume it has crashed. If the job that crashed was in a mapping phase, then other
      TaskTracker nodes will receive requests to re-run all map tasks previously run by the failed node. If the job that crashed was in a reduce phase, then other
      TaskTracker nodes will receive requests to re-run all reduce tasks that were in progress on the failed node.

      Such a simple fault tolerance mechanism is possible because mappers and reducers limit their communication with one another and the outside world. A potential
      drawback with such an approach is that a few slow nodes (called stragglers) can create bottlenecks that hold up the rest of the program. One strategy to remedy
      this problem is known as speculative execution, where the same task is assigned to multiple nodes to decrease the expected time in which it will be finished.

    \subsection{Common Programming Patterns}

      \subsubsection{Counting and Summing}
        The simplest mapper for a counting problem would output a (key, 1) tuple for an instance of a given key. Alternatively, the mapper can aggregate data for an
        entire document, or a combiner can be run just after the mapper to aggregate data across documents processes by a map task in a similar way to how a reducer
        would.

      \subsubsection{Selection}
        Selection returns a subset of input elements that meet a certain requirement. Selection can be handled entirely in the map task; a reducer need not be
        specified, in which case one output file will be generated per map task.

      \subsubsection{Projection}
        Projection returns a subset of fields of each input element (e.g. $a,b,c$ becomes $a,b$). A mapper can handle projection on individual elements, while a
        reducer is only needed to eliminate duplicates.

      \subsubsection{Index Inversion}
        Index inversion produces a mapping of terms to document ids. Mappers emit (term, id) tuples, while reducers combine these tuples to generate lists of ids
        for each term.

      \subsubsection{Cross-Correlation}
        Cross-correlation problems provide as input a set of tuples of items, and for each possible pair of items, the number of items where the tuples co-occur is
        measured. If $N$ items are provided, then $N^2$ values should be reported. This problem can be represented through a matrix if $N^2$ is small enough.
        For larger values of $N^2$, MapReduce can be effective.

        A pairing implementation where mappers return tuples with a pair and a 1 count is simple but often not performant. A ``stripes'' approach where mappers
        return tuples with an item and a list of items it appears with tends to perform better despite requiring more memory for map-side aggregation.

  \section{Apache Spark}

    \subsection{Background}
      Cluster computing frameworks (e.g. Hadoop) allow for large-scale data computations that automatically handle work distribution and fault tolerance. However,
      many of these frameworks do not leverage distributed memory efficiently, making them ineffective for computations with intermediate results. Apache Spark
      introduces the concept of resilient distributed datasets (RDDs), which are parallel, fault-tolerant data structures that persist intermediary results in memory.

    \subsection{Lineage}
      A lineage is a model of the flow of data between RDDs. Spark designs a lineage to perform efficient computations, and also uses the lineage to determine which
      RDDs to rebuild to recover from failures.

    \subsection{Transformations and Actions}
      Transformations are data operations that convert an RDD or a pair of RDDs into another RDD. Actions are data operations that convert an RDD into an output. When
      an action is invoked on an RDD, the Spark scheduler puts together a DAG of transformations.

      \subsubsection{Narrow vs Wide Dependencies}
        The transformations in the DAG are grouped into stages. A single stage is a collection of transformations with narrow dependencies, meaning that one partition of
        the output depends only on a single partition of the input (e.g. map, filter). The boundaries between stages feature wide dependencies, meaning that a single
        partition of output may correspond to multiple partitions of input (e.g. group by key), so the transformations may require a shuffle.

    \subsection{Spark vs Hadoop MapReduce}
      Two common differences between Spark and MapReduce are:
      \begin{itemize}
        \item Spark stores intermediary results in memory, while MapReduce dumps results to HDFS between each job. The MapReduce approach can lead to unnecessary I/O
          when running multiple jobs in sequence.
        \item Spark can support some more complicated workflows within a single job rather than running several new ones.
      \end{itemize}

\end{document}
